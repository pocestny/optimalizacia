\section{\maxflow--\mincut očami duality}

\noindent Dualita lineárneho programovania, ktorú sme predstavili v
predchádzajúcej časti, často pomáha lepšie pochopiť min--max charakterizácie
výpočtových problémov.  V tejto časti si z pohľadu duality priblížime známe
tvrdenie, že veľkosť maximálneho toku sa rovná kapacite minimálneho rezu.


\noindent
Začneme tým, že predstavíme problém maximálneho toku. Keďže predpokladáme, že čitateľ
sa s ním už stretol, odpustíme si tentokrát motivačné rozprávky a zopakujeme len definíciu. 
Majme daný (neorientovaný) graf $G=(V,E)$ s nezápornými váhami hrán, t.j. 
funkciou $c:E\mapsto\R^+$; namiesto $c(\{u,v\})$ budeme skrátene písať $c_{uv}=c_{vu}$.
V grafe sú dva význačné vrcholy $s$ a $t$.
V probléme \maxflow je cieľom nájsť čo najväčší tok z $s$ do $t$. Tok je funkcia 
$f:V^2\mapsto\R^+$, ktorej interpretácia je taká, že $f(u,v)$ je množstvo kvapaliny, ktorá tečie
z $u$ do $v$. Funkcia $f$ musí spĺňať tieto vlastnosti:
\begin{enumerate}
  \item ak $f(u,v)\not=0$, potom $(u,v)\in E$, t.j. tiecť musí po hranách grafu
  \item $f(u,v)=-f(v,u)$, t.j. znamienko udáva, ktorým smerom tok ide
  \item pre každé $v\not\in\{s,t\}$ platí $\sum\limits_{u\in V}f(u,v)=0$, t.j. čo do vrchola vteká, to z neho aj vytečie
\end{enumerate}
\noindent
Veľkosť toku je $\sum_{v\in V}f(s,v)$. Rez v grafe $G$ je množina hrán, ktorej odobratie oddelí vrcholy $s$ a $t$.
Cena rezu je súčet váh prerezaných hrán. Problém \mincut je nájsť rez minimálnej ceny. 
Na nasledovnom obrázku je graf s kapacitami hrán a minimálnym rezom (vľavo) a maximálnym tokom 
(vpravo; tmavo sú označené hrany, ktorých kapacita je naplnená tokom) s hodnotou 24.


\begin{myfig}{\textwidth}{svg/flowcutNew}
\end{myfig}


\noindent
Problém \maxflow sa dá prirodzene formulovať ako úloha lineárneho programovania, a to hneď niekoľkými
spôsobmi. Z dôvodov, ktoré, ako dúfame, budú jasné na konci tejto časti, si zvolíme takúto 
formuláciu: pre každú hranu $(u,v)\in E$ budeme mať dve nezáporné premenné $x_{uv}$ a $x_{vu}$, 
ktoré budú udávať množstvo toku z $u$ do $v$ (všimnite si, že sme tým povolili situáciu, 
že po tej istej hrane tečie tok oboma smermi: $x_{uv}$ v smere z $u$ do $v$ a $x_{vu}$ v
smere z $v$ do $u$; ničmenej, tieto dve hodnoty stačí odčítať a dostaneme riešenie
v zmysle našej definície). Ďalej budeme mať jednu premennú $f$, ktorá bude udávať veľkosť
toku (a tú sa budeme snažiť maximalizovať): $f=\sum\limits_{u:(s,u)\in E}x_{su}-\sum\limits_{u:(s,u)\in E}x_{us}$. 
Zákony zachovania toku aj obmedzenia kapacít  
ľahko zapíšeme pomocou lineárnych nerovností a dostaneme program

\begin{equation}
  \label{eq:flow:1}
  \begin{array}{rrcll}
      {\rm maximalizovať}     & \multicolumn{1}{r}{f}\\[8mm]
    {\rm pri\ obmedzeniach} & \sum\limits_{u:(s,u)\in E} x_{su} - \sum\limits_{u:(s,u)\in E} x_{us} - f  &=&0&  \\[6mm]
                            & \sum\limits_{u:(t,u)\in E} x_{ut} - \sum\limits_{u:(s,u)\in E} x_{tu} + f  &=&0&  \\[6mm]
                            & \sum\limits_{u:(u,v)\in E} x_{vu} - \sum\limits_{u:(u,v)\in E} x_{uv} &=&0&\;\;\;
    \forall v\in V - \{s,t\}\\[6mm]
    & x_{uv}&\le& c_{uv}&  \;\;\;\forall (u,v)\in E\\[1mm]
    & x_{vu}&\le& c_{uv}&  \;\;\;\forall (u,v)\in E\\[1mm]
                            & x_{uv}&\ge& 0 &  \;\;\;\forall (u,v)\in E\\
  \end{array}
\end{equation}  

\noindent
Prvé a druhé obmedzenie definujú veľkosť toku $f$ ako to, čo vyteká z $s$, resp. čo vteká do $t$. Vo všetkých
ostatných vrcholoch platí zákon zachovania toku. 

\noindent
Použime teraz náš dualizačný recept a zostrojme k programu (\ref{eq:flow:1}) duálny minimalizačný program. 
Každému obmedzeniu v primárnom programe zodpovedá premenná v duálnom programe.
V programe (\ref{eq:flow:1}) máme dva druhy obmedzení: obmedzenia v tvare $\cdots = 0$ pre každý vrchol
a obmedzenia $x_{uv}\le c_{uv}$. Zavedieme si preto duálne premenné $y_v\in\R$ pre každý vrchol $v\in V$
($y_s$ zodpovedá prvému obmedzeniu, $y_t$ druhému a ostatné $y_v$ v poradí ďalším)
a premenné $z_{uv}$ a $z_{vu}$ pre každú hranu $(u,v)\in E$, pričom $z_{uv},z_{vu}\ge0$.
Duálny program sa vyrába tak, že $i$-ta duálna premenná je  multiplikátor, ktorým prenásobíme
$i$-te obmedzenie a výsledky sčítame; súčet pravých strán je príslušný odhad a vyžadujeme, aby
súčet ľavých strán bol v každej zložke väčší ako maximalizovaná funkcia.
Súčet pravých strán je v našom prípade $\sum\limits_{(u,v)\in E}(z_{uv}+z_{vu})c_{uv}$.
Maximalizovaná funkcia má pri všetkých premenných okrem $f$ hodnotu 0. 
Premenná $f$ vystupuje iba v prvom a druhom obmedzení (\ref{eq:flow:1}), preto dostávame
v duálnom programe obmedzenie $y_t-y_s=1$ (je jasné, že tok je nezáporný, ale formálne sme 
v (\ref{eq:flow:1}) nevyžadovali, aby $f\ge0$). Každá premenná $x_{uv}$ sa vyskytuje v 
dtroch obmedzeniach: v obmedzení zodpovedajúcom vrcholu $u$ s kladným znamienkom,  v obmedzení
zodpovedajúcom vrcholu $v$ so záporným znamienkom a v obmedzení $x_{uv}\le c_{uv}$. Dostávame
tak duálny program


\begin{equation}
  \label{eq:flow:2}
  \begin{array}{rrcll}
    {\rm minimalizovať}     & \multicolumn{1}{r}{\sum\limits_{(u,v)\in E}(z_{uv}+z_{vu})\,c_{uv}}\\[8mm]
    {\rm pri\ obmedzeniach} & y_t - y_s  &=&1&  \\[3mm]
                            & y_v - y_u + z_{uv} &\ge&0&\;\;\;    \forall (u,v)\in E \\[3mm]
                            & y_u - y_v + z_{vu} &\ge&0&\;\;\;    \forall (u,v)\in E \\[3mm]
                            & z_{uv},z_{vu}&\ge& 0 &  \;\;\;\forall (u,v)\in E\\
  \end{array}
\end{equation}  

\noindent
Z vety o dualite vieme, že programy (\ref{eq:flow:1}) a (\ref{eq:flow:2}) majú rovnakú hodnotu optima. 
Ako môžme program (\ref{eq:flow:2}) interpretovať? 
Bez ujmy na všeobecnosti môžme predpokladať, že aspoň jedna z dvojice premenných $z_{uv},z_{vu}$
je nulová: jediné obmedzenia na $z_{uv}$ a $z_{vu}$
sú $z_{uv}\ge y_u-y_v$ a $z_{vu}\ge y_v-y_u$. Zjavne aspoň jedna z hodnôt $y_u-y_v$ a $y_v-y_u$
nie je kladná, a teda ak máme ľubovoľné prípustné riešenie, tak aspoň jednu z premenných 
$z_{uv},z_{vu}$ môžme nastaviť na 0 a nezväčšíme hodnotu minimalizovanej funkcie. 
Môžme si preto označiť $\bar{z}_{uv}=\max\{z_{uv},z_{vu}\}$; obmedzenia potom hovoria, že
$\bar{z}_{uv}\ge|y_u-y_v|$ a z rovnakých dôvodov ako pred chvíľou môžeme bez ujmy na všeobecnosti
predpokladať, že $\bar{z}_{uv}=|y_u-y_v|$. Môžme si predstaviť, že vrcholy grafu ukladáme 
na priamku: vrchol $v$ je uložený v bode $y_v$, pričom, opäť bez ujmy na všeobecnosti, je $s$ uložený
v bode 0 a $t$ v bode 1. $\bar{z}_{uv}$ je dĺžka hrany $(u,v)$ v našom uložení.
Keď to zhrnieme, optimálne riešenie programu (\ref{eq:flow:2}) nám dá také uloženie vrcholov grafu na 
úsečku dĺžky 1, že $s$ a $t$
sú na krajoch a minimalizuje sa váhovaná dĺžka hrán.

\begin{myfig}{0.9\textwidth}{svg/relaxcut}
  Graf s váhami hrán a jedno z možných optimálnych riešení programu (\ref{eq:flow:2}):
  $\bar{z}_{su}=\bar{z}_{vw}=\frac{1}{6}$, $\bar{z}_{sv}=\bar{z}_{uw}=\frac{1}{3}$,
  $\bar{z}_{sw}=\bar{z}_{wt}=\frac{1}{2}$. Výsledná hodnota je
  $$\sum_{(u,v)\in E}c_{uv}\bar{z}_{uv}=\frac{1}{6}+\frac{1}{3}+2\frac{1}{2}+\frac{1}{3}+\frac{1}{6}+4\frac{1}{2}=4.$$
\end{myfig}

\noindent
Poďme teraz prepísať program (\ref{eq:flow:2}) do normálneho tvaru 
$\max\{-\bm{c}\tr\bm{\beta}\mid A\bm{\beta}=0,\;\bm{\beta}\ge0\}$.
Ku každej premennej $z_{uv}$ zavedieme premennú $\hat{z}_{uv}\ge 0$, aby sme získali
obmedzenia tvaru \hbox{$y_v-y_u+z_{uv}-\hat{z}_{uv}=0$}.
Ak vektor  $\bm{\beta}$  pozostáva zaradom z hodnôt 
$y_s,y_t,y_{v_1},\ldots,z_{uv},z_{vu},\ldots,\hat{z}_{uv}.\hat{z}_{vu},\ldots$, 
matica $A$ má takúto štruktúru:

\begin{myfig}{0.8\textwidth}{svg/cutlp}
\end{myfig}

\noindent
Nasledujúcu priamočiaru povinnú jazdu prenecháme na čitateľa:

\begin{prob}
  S pomocou viet \ref{thm:idTum} a \ref{thm:unimod} ukážte, že matica $A$ je TUM.
\end{prob}


\noindent
Z Vety~\ref{thm:tumInteger} vyplýva, že existuje celočíselné optimálne riešenie programu (\ref{eq:flow:2}).
To znamená, že všetky vrcholy majú $y_v\in\{0,1\}$ (t.j. sú uložené na niektorom konci úsečky)
a každá hrana má dĺžku 0 alebo 1. Preto každá cesta z $s$ do $t$ musí obsahovať aspoň jednu hranu s dĺžkou $1$,
a teda ak hrany dĺžky 1 z grafu odstránime, dostaneme rez, ktorý oddeľuje $s$ od $t$. Môžme teda povedať

\begin{veta}[\maxflow-\mincut veta]
  Veľkosť maximálneho toku sa rovná kapacite minimálneho rezu.
\end{veta}
\begin{dokaz}
  Veľkosť maximálneho toku je optimum programu (\ref{eq:flow:1}), ktoré je rovnaké, ako optimum
  programu (\ref{eq:flow:2}). Existuje celočíselné optimum programu (\ref{eq:flow:2}), a zároveň
  je bijekcia medzi $s-t$ rezmi a celočíselnými riešeniami  (\ref{eq:flow:2}): celočíselnému riešeniu
  (\ref{eq:flow:2}) prislúcha rez a každému rezu vieme prirodzeným spôsobom priradiť riešenie  (\ref{eq:flow:2}).
\end{dokaz}

\noindent
Je dosť možné, že čitateľ možno pozná oveľa jednoduchší dôkaz tohto tvrdenia. Dôvod, prečo uvádzame tento dôkaz je
(okrem toho, že chceme ilustrovať dualitu lineárnych programov) ten, že ukazuje dvojicu 
\maxflow-\mincut ako špeciálny prípad všeobecnejších problémov; zároveň dáva nový pohľad na otázku, prečo
rovnaký výsledok neplatí pre iné podobné problémy (ak napríklad máme viacero zdrojov a ústí a cieľom je
maximalizovať tok rôznych komodít v spoločných potrubiach, máme duálany problém k problému
\minmulticut z Definície~\ref{dfn:multicut}, avšak analogický výsledok neplatí) cez unimodularitu príslušných matíc.

