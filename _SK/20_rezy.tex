V doterajšom texte sme si predstavili rôzne spôsoby, ako využiť optimalizačné metódy pri návrhu aproximačných
algoritmov. V nasledujúcich dvoch kapitolách si priblížime niektoré známe problémy a \ldots


Problém nájdenia minimálneho rezu v grafe, \mincut, by mal byť čitateľom dobre známy:

\begin{framed}
  \begin{dfn}
    \label{dfn:mincut}
    Majme daný jednoduchý graf $G=(V,E)$ s hranami ohodnotenými nezápornými váhami, t.j. funkciou
    $\omega:E\mapsto \R^+$ a v ňom dvojicu vrcholov $s,t$. Cieľom problému
    \mincut je odobrať z grafu $G$ množinu hrán s minimálnou celkovou váhou tak, aby
     vrcholy $s$ a $t$ neboli v rovnakom komponente súvislosti výsledného grafu.
  \end{dfn}
\end{framed}

Zaujímavá je spojitosť \mincut a \maxflow: známa veta hovorí, že veľkosti minimálneho rezu a maximálneho toku 
sú rovnaké. 


\begin{myfig}{\textwidth}{svg/flowcutNew}
  \centerline{Minimálny rez a maximálny tok sú rovnaké.}
\end{myfig}

Videli sme, že \maxflow sa dá vyjadriť ako lineárny program, ktorého duálny program
je relaxácia problému \mincut (v ktorej nevyberáme množinu hrán, ale každej hrane priradíme hodnotu
z intervalu $[0,1]$ a požadujeme, aby na každej $s-t$ ceste bol súčet hodnôt aspoň 1).
Zároveň, pretože matica obmedzení tohto relaxovaného programu je totálne unimodulárna, vieme,
že existuje celočíselné optimum. Efektívnu riešiteľnosť \mincut môžeme teda pripísať na vrub
špeciálnemu tvaru matice príslušného lineárneho programu. Dalo by sa očakávať, 
že programy takéhoto tvaru sú skôr výnimkou a pre väčšinu problémov podobná min-max charakterizácia
neplatí. V tomto texte chceme ilustrovať, že efektívne riešiteľné 
diskrétne optimalizačné problémy sú skôr výnimka ako pravidlo.
Prezentujeme niekoľko zovšeobecnení \mincut, ktoré sú \NP-ťažké
a ukážeme aproximačné algoritmy na ich riešenie.
Jedno zovšeobecnenie, v ktorom namiesto jednej dvojice vrcholov treba rozpojiť $k$ dvojíc, sme už videli:


\begin{framed}
  \begin{dfn}
    \label{dfn:multicut}
    Majme daný jednoduchý graf $G=(V,E)$ s hranami ohodnotenými nezápornými váhami, t.j. funkciou
    $\omega:E\mapsto \R^+$ a v ňom $k$ dvojíc vrcholov $(s_i,t_i)$, $i=1,\ldots,k$. Cieľom problému
    \minmulticut je odobrať z grafu $G$ množinu hrán s minimálnou celkovou váhou tak, aby
    žiadna dvojica $(s_i,t_i)$ nebola v rovnakom komponente súvislosti výsledného grafu.
  \end{dfn}
\end{framed}

a ukázali sme $4\ln(2k)$-aproximačný algoritmus. 

%%%%%%%%%%%%%%%%%%%%%%%%%%%%%%%%%%%%%%%%%%%%%%%%%%%%%%%%%%%%%%%%%%%%%%%%%%%%%%%%%%%%%%%%%%%%%%%%%%%%%%%%%%%%
\section*{Variácia prvá: \multiwaycut}

Iné zovšeobecnenie nepožaduje rozpojiť konkrétne dvojice vrcholov, ale iba rozdeliť 
niekoľko daných význačných vrcholov (terminálov) do rôznych komponentov súvislosti:

\begin{framed}
  \begin{dfn}
    \label{dfn:multiwaycut}
    Majme daný jednoduchý graf $G=(V,E)$ s hranami ohodnotenými nezápornými váhami, t.j. funkciou
    $\omega:E\mapsto \R^+$ a v ňom $k$  vrcholov $s_1,\ldots,s_k$. Cieľom problému
    \multiwaycut je odobrať z grafu $G$ množinu hrán s minimálnou celkovou váhou tak, aby
    žiadne dva vrcholy $s_i$, $s_j$  neboli v rovnakom komponente súvislosti výsledného grafu.
  \end{dfn}
\end{framed}

Bez ujmy na všeobecnosti môžeme predpokladať, že vstupný graf je súvislý (inak každý komponent
súvislosti môžme vyriešiť ako samostatnú inštanciu), a preto optimálne riešenie bude mať
práve $k$ komponentov (menej nemôže mať, lebo to by museli dva terminály byť v jednom komponente,
a viac nebude mať kvôli minimalite). 

\begin{myfig}{.6\textwidth}{svg/multiway1}
    Optimálne riešenie s cenou $43$. Platí
    $\partial(C_1)=21$,
    $\partial(C_2)=22$,
    $\partial(C_3)=1$,
    $\partial(C_4)=21$,
    $\partial(C_5)=1$,
  $\partial(C_6)=20$.
\end{myfig}


Označme $C_i$ komponent súvislosti optimálneho riešenia, ktorý obsahuje terminál $s_i$; nech
$\partial(C_i)$ je veľkosť rezu, ktorý oddeľuje komponent $C_i$, t.j.
\begin{equation}
  \label{eqn:cutedgeboundary}
  \partial(C_i):=\sum_{e=(u,v)\atop u\in C_i, v\not\in C_i} \omega(e).
\end{equation}

Cena optimálneho riešenia je potom 
\begin{equation}
  \label{eqn:multiwaycut:1}
  OPT=\frac{1}{2}\sum_{i=1}^k\partial(C_i),
\end{equation}

lebo každá hrana rezu je v sume na pravej strane zarátaná dvakrát.
Ukážeme si $2(1-\frac{1}{k})$-aproximačný algoritmus, t.j. algoritmus, ktorý vždy
nájde rez s hodnotou najviac $2(1-\frac{1}{k})$-násobku optima. 

Zoberme si ľubovoľný terminál $s_i$. 
Označme $D_i$ minimálny rez, ktorý oddeľuje $s_i$ 
od ostatných terminálov a označme $\partial(D_i)$ jeho veľkosť. 
Rez $D_i$ vieme pre každé $i$ vypočítať ľahko: všetky ostatné terminály zlúčime do jedného
vrchola a v takto získanom grafe zrátame minimálny rez.

\begin{minipage}[t]{0.45\textwidth}
  \vskip 0pt
\begin{myfig}{\textwidth}{svg/multiway2}
  \centerline{Rez $D_2$ má veľkosť 22.}
\end{myfig}
\end{minipage}
\hfill
\begin{minipage}[t]{0.45\textwidth}
  \vskip 0pt
\begin{myfig}{\textwidth}{svg/multiway3}
  Veľkosti rezov sú 
    $\partial(D_1)=20$,
    $\partial(D_2)=22$,
    $\partial(D_3)=1$,
    $\partial(D_4)=19$,
    $\partial(D_5)=1$,
    $\partial(D_6)=19$. Cena riešenia je $80$.
\end{myfig}
\end{minipage}

Čo sa stane, ak  za riešenie zoberieme zjednotenie rezov $D_1\cup D_2\cup\cdots\cup D_k$?
Každý rez $C_i$ z optimálneho riešenia oddeľuje $s_i$ od zvyšných terminálov. Keďže $D_i$ je minimálny
taký rez, je $\partial(D_i)\le\partial(C_i)$.
Zároveň pre cenu riešenia $m$ platí 
$$m\le\sum_{i=1}^k\partial(D_i)\le\sum_{i=1}^k\partial(C_i)=2\cdot OPT,$$
kde posledná rovnosť je (\ref{eqn:multiwaycut:1}), a preto máme 2-aproximačný algoritmus. Zároveň hneď vidíme,
ako ho trochu vylepšiť: prípustné riešenie dostaneme, ak urobíme zjednotenie hociktorých $k-1$ rezov  $D_i$ namiesto
všetkých $k$. Prirodzenou voľbou je vynechať ten najťažší; nech je to $D_j$. Keďže je najťažší, z Dirichletovho
princípu vyplýva, že musí byť aspoň tak ťažký ako priemer, t.j.
$$\partial(D_j)\ge\frac{1}{k}\sum_{i=1}^k\partial(D_i).$$
Pre cenu riešenia výsledného algoritmu potom dostávame
$$ALG\le\sum_{i=1\atop i\not=j}^k\partial(D_i)=\sum_{i=1}^k\partial(D_i)-\partial(D_j)\le
\left(1-\frac{1}{k}\right)\sum_{i=1}^k\partial(D_i)\le
\left(1-\frac{1}{k}\right)\sum_{i=1}^k\partial(C_i)=2\left(1-\frac{1}{k}\right) OPT.$$


%%%%%%%%%%%%%%%%%%%%%%%%%%%%%%%%%%%%%%%%%%%%%%%%%%%%%%%%%%%%%%%%%%%%%%%%%%%%%%%%%%%%%%%%%%%%%%%%%%%%%%%%%%%%
\section*{Variácia druhá: \kcut}

Poslednou variáciou v tomto texte je modifikácia problému \multiwaycut, pri ktorej nemáme žiadne terminály 
a chceme iba graf rozdeliť na (aspoň) $k$ častí:

\begin{framed}
  \begin{dfn}
    \label{dfn:kcut}
    Majme daný jednoduchý graf $G=(V,E)$ s hranami ohodnotenými nezápornými váhami, t.j. funkciou
    $\omega:E\mapsto \R^+$ a číslo $k$.
     Cieľom problému
    \kcut je odobrať z grafu $G$ množinu hrán s minimálnou celkovou váhou tak, aby
    výsledný graf mal aspoň $k$ komponentov súvislosti.
  \end{dfn}
\end{framed}

Podotýkame, že $k$ v názve problému je len symbol; počet častí, na ktoré treba rozkrájať graf, je súčasťou vstupu.
Opäť ide o \NP-ťažký problém a ukážeme si algoritmus s rovnakou garanciou aproximácie ako v predchádzajúcom prípade, 
t.j. $2\left(1-\frac{1}{k}\right)$. Predstavíme pri tom aj dátovú štruktúru, ktorá môže byť zaujímavá sama osebe.

\subsection*{Gomory-Hu stromy}

Predstavme si situáciu, že máme graf $G=(V,E)$ 
s hranami ohodnotenými nezápornými váhami  
a chceme pre každú dvojicu vrcholov $u,v$ zrátať cenu minimálneho $u-v$ rezu.
Priamočiary prístup je $\Omega(n^2)$-krát zavolať procedúru na výpočet minimálneho rezu.
Dá sa to ale aj lepšie: v skutočnosti nám stačí $O(n)$ počítaní minimálneho rezu. Kľúčom je
dátová štruktúra, ktorá efektívne kóduje minimálne rezy medzi všetkými dvojicami vrcholov, tzv.
Gomory-Hu strom. 

\vskip 1ex

V ďalšom budeme používať nasledovné označenie: 

\begin{itemize}
  \item Pre dva vrcholy $u,v\in V$, $f_G(u,v)$ (alebo
len $f(u,v)$ ak je $G$ zrejmé z kontextu), bude veľkosť minimálneho $u-v$ rezu v $G$. 
\item Ďalej,
nech $T=(V,E)$ je strom a $e\in E$ je hrana. Po jej odstránení dostaneme graf $T-\{e\}$, ktorý
má dva komponenty súvislosti. Označme $cut_T(e)\subseteq V$  
jeden\footnote{Aby sme mali jednoznačnú definíciu, potrebujeme povedať, ktorý komponent vyberieme.
  Keďže ale budeme hovoriť o veľkosti rezu medzi komponentami, 
  je nám to jedno. Napríklad nech $cut_T(e)$ je menší z komponentov $T-\{e\}$
  a v prípade rovnosti $cut_T(e)$ je 
ten komponent $T-\{e\}$, ktorý obsahuje nejaký pevne zvolený fixný vrchol $v_0$.}
z nich. 
\item Nech $G=(V,E)$ je graf. Tak ako v (\ref{eqn:cutedgeboundary}), 
  pre množinu $S\subseteq V$ bude $\partial_G(S)$ (resp. $\partial(S)$, ak je $G$ zrejmé)
veľkosť rezu určeného množinou $S$
(zjavne $\partial(S)=\partial(V-S)$).
\end{itemize}

\begin{framed}
  \begin{dfn}
    \label{dfn:GomoryHu}
    Majme daný jednoduchý graf $G=(V,E)$ s hranami ohodnotenými nezápornými váhami, t.j. funkciou
    $\omega:E\mapsto \R^+$. {\em Gomory-Hu strom} ku grafu $G$ je strom $T=(V,E')$, s hranami
    ohodnotenými funkciou  $\omega':E'\mapsto \R^+$, ktorý spĺňa tieto vlastnosti:
    \begin{enumerate}
      \item $\forall e'\in E':\;\omega'(e')=\partial_G(cut_T(e'))$
      \item $\forall u,v\in V:\;f_G(u,v)=f_T(u,v)$
    \end{enumerate}
  \end{dfn}
\end{framed}

Definícia~\ref{dfn:GomoryHu} nehovorí, ako taký strom skonštruovať (ani nehovorí, či je jednoznačný), ale iba
to, že každý strom s danými vlastnosťami je Gomory-Hu strom. Čo vlastne v Definícii~\ref{dfn:GomoryHu} požadujeme?
Strom $T$ má rovnakú množinu vrcholov ako $G$, ale hrany môže mať úplne iné (nijak nesúvisia s hranami $G$).
Prvá vlastnosť hovorí, že ak by sme už poznali hrany $E'$, ich váhy $\omega'$ vyrátame ľahko: pre každú
hranu $e'$ sa pozrieme, na aké množiny sa rozpadne $T$ po odobratí $e'$, a potom zrátame veľkosť
rezu v $G$ medzi týmito množinami. Napríklad na nasledujúcom obrázku je $cut_T((j,c))=\{c,d\}$
a $\partial_G(\{c,d\})=20$ (z množiny $\{c,d\}$ odchádzajú hrany $(b,c)$ a $(e,c)$).


\begin{myfig}{\textwidth}{svg/kcut1}
  \centerline{Graf $G$ a jeho Gomory-Hu strom $T$.}
\end{myfig}


Druhá vlastnosť hovorí, že z $T$ sa dajú vyčítať hodnoty minimálneho rezu medzi ľubovoľnými dvoma vrcholmi:
$f_G(u,v)$, teda veľkosť minimálneho $u-v$ rezu v $G$, je $f_T(u,v)$: $T$ je strom, takže obsahuje práve jednu
$u-v$ cestu, a preto $f_T(u,v)$ je minimálna váha $\omega'$ na tejto ceste. Napríklad pre vrcholy $a,k$ na obrázku
je v $T$ cesta $a-b-j-k$ a $f_T(a,k)=20$. Minimum sa nadobúda na hrane $a-b$ a $cut_T((a,b))=\{a\}$. Vskutku,
rez $\{a\}$ s $\partial_G(\{a\})=20$ (kvôli hranám $(a,b)$, $(a,j)$) je minimálny $a-k$ rez v $G$.

\vskip 2ex

Poďme teraz ukázať algoritmus, ktorý vyrobí k danému grafu $G$ jeho Gomory-Hu strom. 
Bude postupovať v iteráciách, pričom v každej iterácii $t$ bude mať
partíciu vrcholov $V=S^{(t)}_1\cup S^{(t)}_2\cup\cdots\cup S^{(t)}_{n_t}$, 
pričom $S^{(t)}_i\cap S^{(t)}_j=\emptyset$ 
pre $i\not=j$. Množiny $S^{(t)}_i$ budeme volať krabice. Na začiatku sú všetky vrcholy v jednej krabici,
t.j. $n_0=1$, $S^{(0)}_1=V$. 
Na konci chceme, aby každá krabica obsahovala jeden vrchol; po skončení algoritmu môžeme preto
jednoprvkové krabice stotožniť s príslušnými vrcholmi.

Na začiatku iterácie $t+1$ sú krabice
pospájané stromom $T^{(t)}=\left(\left\{S_i^{(t)}\right\}_{i=1}^{n_t},E^{(t)}\right)$ s ohodnotenými hranami. Iterácia $t+1$
zoberie jednu krabicu $S=S^{(t)}_i$ a zakorení $T^{(t)}$ v $S$. Vyberie z $S$ dva vrcholy $x,y$
a rozdelí $S$ na dve krabice $S_x$ a $S_y$ spojené hranou, pričom zvolí váhu pre novú hranu a
vhodne prerozdelí podstromy.
Cieľom je implementovať rozdeľovanie krabíc a prerozdeľovanie podstromov tak, aby po skončení algoritmu bol strom $T$
Gomory-Hu strom pre graf $G$.

\begin{myfig}{0.75\textwidth}{svg/kcut2}
\end{myfig}

Počas behu algoritmu bude platiť nasledovný invariant:

\begin{framed}
{\bf Invariant:} Nech $e\in E^{(t)}$ je ľubovoľná hrana v strome $T^{(t)}$, $e=\left(S_i^{(t)},S_J^{(t)}\right)$,
potom $cut_{T^{(t)}}(e)$ je množina krabíc v jednom komponente $T^{(t)}-\{e\}$. Nech $M$
sú vrcholy grafu $G$ z týchto krabíc, t.j. $M:=\{v\in V\mid \exists S\in cut_{T^{(t)}}(e):\;v\in S\}$.
Potom existujú dvaja {\em svedkovia} $x\in S_i^{(t)}$, $y\in S_j^{(t)}$, že $\omega'(e)=f_G(x,y)$ a 
$M$ je minimálny $x-y$ rez v $G$.
\end{framed}

Inými slovami, keď si zoberieme hocijakú hranu $e$ zo stromu krabíc, vieme jej priradiť rez v grafe $G$: rez je
definovaný množinou tých vrcholov z $V$, ktoré sú v niektorej krabici z jedného komponentu $T^{(t)}-\{e\}$.
Tento rez v $G$ musí byť minimálny $x-y$ rez pre svedkov $x,y$, a zároveň cena hrany $e$ v strome $T^{(t)}$
musí byť cena tohoto rezu.

Teraz potrebujeme urobiť dve veci: jednak ukázať, že keď algoritmus skončí a platí invariant, máme dobrý 
Gomory-Hu strom a dvak navrhnúť iteráciu algoritmu tak, aby invariant ostával v platnosti.

Najprv si ukážeme, že ak platí invariant a každá krabica je jednoprvková, máme Gomory-Hu strom. 
Nech $T=(V,E')$ je strom po skončení algoritmu a nech v ňom platí invariant. Keďže každá krabica bola jednoprvková,
svedkovia z invariantu sú samotné vrcholy.
Potrebujeme ukázať
obe vlastnosti z Definície~\ref{dfn:GomoryHu}. Prvá vlastnosť, $\omega'(e)=\partial_G(cut_T(e))$,  hovorí, 
že váha hrany $e\in E'$ je cena príslušného rezu v $G$. Z invariantu ale platí, že $cut_T(e)$ je minimálny
$x-y$ rez, kde $e=(x,y)$. Zároveň invariant hovorí, že $\omega'(e)=f_G(x,y)=\partial_G(cut_T(e))$.
Prvá vlastnosť z Definície~\ref{dfn:GomoryHu} je preto splnená. 

Druhá vlastnosť hovorí, že $f_G(u,v)=f_T(u,v)$
pre ľubovoľné dva vrcholy $u,v\in V$. Ak $(u,v)\in E'$, vlastnosť vyplýva priamo z invariantu:
keďže $u,v$ sú spojené hranou v strome $T$, je $f_T(u,v)=\omega'((u,v))=f_G(u,v)$.
Nech teda $u,v$ nie sú spojené hranou v $T$. Keďže $T$ je strom, je v ňom práve jedna $u-v$ cesta
$u=w_0,w_1,\ldots,w_z=v$ a minimálny $u-v$ rez v $T$ je hrana s minimálnou váhou na nej.
Označme túto hranu $e_{\min}=(w_i,w_{i+1})$, t.j. chceme ukázať $f_G(u,v)=\omega'(e_{\min})$.

Na jednej strane, v $T-\{e_{\min}\}$ sú $w_{i}$ a $w_{i+1}$ v rôznych komponentoch, a teda aj $u$ a $v$
sú v rôznych komponentoch. Preto $cut_T(e_{\min})$ je rez v $G$, ktorý oddelí $u$
od $v$, a teda $\partial_G(cut_T(e_{\min}))\ge f_G(u,v)$. Z predchádzajúcich úvah ale vieme,
že $\partial_G(cut_T(e_{\min}))=\omega'(e_{\min})$, a tak sme ukázali, že $\omega'(e_{\min})\ge  f_G(u,v)$.

\begin{myfig}{\textwidth}{svg/kcut3}
  Vľavo je graf $G=(V,E)$, vpravo jeho Gomory-Hu strom $T$. Medzi vrcholmi $u$, $v$ je v $T$ jediná cesta
a na nej je hrana $e_{\min}$. Odstránením hrany $e_{\min}$ z $T$ dostaneme rozklad $V$ na dve množiny,
a teda rez v $G$. Vieme, že cena hrany $e_{\min}$ v $T$ je veľkosť minimálneho rezu $G$ medzi
koncovými vrcholmi $e_{\min}$. Teraz sa pokúšame ukázať, že tento rez je zároveň minimálny $u-v$ rez.
\end{myfig}


Na to, aby sme ukázali opačnú nerovnosť, t.j. $\omega'(e_{\min})\le  f_G(u,v)$, si pomôžeme nasledovnou lemou:

\begin{lema}
  \label{lm:kcutreclema}
  Majme graf $G=(V,E)$ a nech $\{v_1,v_2,\ldots,v_z\}\subseteq V$. Potom
  $$f_G(v_1,v_z)\ge\min\{f_G(v_1,v_2),f_G(v_2,v_3),\ldots,f_G(v_{z-1},v_z)\}.$$
\end{lema}

\begin{dokaz}
  Urobíme indukciu na $z$. Pre $z=2$ lema triviálne platí. Ak $z>2$, z indukčného predpokladu vieme,
  že $f(v_2,v_z)\ge\min\{f(v_2,v_3),\ldots,f(v_{z-1},v_z)\}$. Preto nám stačí ukázať, že
  $f(v_1,v_z)\ge\min\{ f(v_1,v_2), f(v_2,v_z) \}$. Predpokladajme sporom, že 
  $f(v_1,v_z)<\min\{ f(v_1,v_2), f(v_2,v_z) \}$ a nech $C$ je minimálny $v_1-v_z$ rez v $G$. 
  Bez ujmy na všeobecnosti, nech
   $v_2$
   je na rovnakej strane rezu, ako $v_1$ (ináč premenujeme $v_1$ a $v_z$ a máme symetrickú situáciu).
  $C$ je zároveň $v_2-v_z$ rez, a preto
  $f(v_2,v_z)\le f(v_1,v_z)$, spor.
\end{dokaz}

Teraz vieme, že $f_G(u,v)\ge\min\{f_G(w_0,w_1),f_G(w_1,w_2),\ldots,f_G(v_{z-1},w_z)\}.$
Zároveň, pretože $(w_i,w_{i+1})\in E'$, z predchádzajúcich úvah vieme, že 
$f_G(w_i,w_{i+1})=\omega'((w_i,w_{i+1}))$, a teda
$$f_G(u,v)\ge\min\{\omega'(w_0,w_1),\ldots,\omega'(w_{z-1},w_z)\}=\omega'(e_{\min}).$$

\vskip 2ex
Teraz, keď už vieme, že ak po skončení algoritmu invariant platí, tak máme dobrý Gomory-Hu strom,
poďme navrhnúť algoritmus tak, aby invariant ostával v platnosti. Ako sme už povedali, v jednej iterácii
si algoritmus vyberie krabicu $S$ (ľubovoľnú) s aspoň dvoma vrcholmi $x$ a $y$ (ľubovoľnými) a rozdelí
$S$ na dve menšie krabice. Toto rozdelenie sa urobí takto: zakoreníme strom $T^{(t)}$ v  $S$ 
 a nech
$T_1,\ldots,T_z$ sú podstromy synov $S$. Pre podstrom $T_i$ označme $V_i$ tie vrcholy grafu $G$, ktoré
sú v niektorej krabici z tohto podstromu, t.j. $V_i:=\{v\in V\mid \exists S'\in T_i:\;v\in S'\}$.
Z $G$ zostrojíme graf $G'$ tak, že vrcholy z $V_i$ sa stotožnia a nahradia sa novým vrcholom $y_i$,
pričom hrany ostanú (t.j. nový graf môže mať násobné hrany). V grafe $G'$ 
nájdeme minimálny $x-y$ rez $T$. Krabicu $S$ nahradíme dvoma krabicami $S_x:=S\cap T$, $S_y:= S-S_x$.
Do stromu pridáme hranu spájajúcu $S_x$ a $S_y$ s cenou $\partial_G(T)$. Podstromy $T_i$, pre ktoré
$y_i\in T$, budú susediť s $S_x$, zvyšné podstromy s $S_y$.


\begin{myfig}{\textwidth}{svg/kcut4}
  Vľavo je stav na začiatku iterácie: máme štyri krabice pospájané hranami do stromu (modré
  čísla sú váhy $\omega'$ v $T^{(t)}$, čierne sú pôvodné váhy $\omega$ v $G$).
  Jednu krabicu $S$ sme vybrali za koreň a vybrali dva ľubovoľné vrcholy $x,y\in S$.
  $S$ má dvoch synov a príslušné podstromy skontrahujeme do dvoch vrcholov; dostaneme tak graf vpravo,
  v ktorom nájdeme minimálny $x-y$ rez $T$. 
\end{myfig}



\begin{myfig}{\textwidth}{svg/kcut5}
Nájdený rez jednak definuje, ako rozdeliť $S$ na $S_x$ a $S_y$, a zároveň určuje, ktorý podstrom bude kam patriť.
\end{myfig}

Na záver rozprávania o Gomory-Hu stromoch potrebujeme ukázať, že takto definovaný spôsob rozbíjania krabíc
zachováva v platnosti invariant, teda že ak na začiatku iterácie invariant platil, bude platiť aj po jej skončení.
Zjavne pre hrany stromu, ktoré sú v niektorom podstrome $T_i$, sa počas iterácie nič nezmenilo: svedkovia ostali,
váha hrany aj rez ňou definovaný sa taktiež nezmenili. Pre hrany, ktoré v pôvodnom strome boli incidentné s $S$
sa nezmenila váha, ani hodnota rezu, ale mohlo sa stať, že po rozdelení $S$ sa jeden svedok stratil a budeme
musieť nájsť iného. Napokon
treba ukázať platnosť invariantu pre novú hranu $(S_x,S_y)$. 
V tom všetkom nám viackrát pomôže nasledovná lema:

\begin{lema}
  \label{lm:kcutlema}
  Majme graf $G=(V,E)$ a nech $S\subseteq V$ je nejaký minimálny $r-s$ rez pre $r,s\in V$, pričom $s\in S$.
  Ďalej, nech $v,w\in S$ sú dva ľubovoľné vrcholy. Potom existuje minimálny $v-w$ rez $T$ v $G$ taký, že 
  $T\subset S$.
\end{lema}

\begin{dokaz}
  Zoberme si nejaký minimálny $v-w$ rez $X$. Ak $X\subset S$ (alebo $V-T\subset S$), niet čo dokazovať.
  Takže predpokladajme, že $S\cap X\not=\emptyset$ a uvažujme výraz $A:=\partial(S)+\partial(X)$. 
  Podľa príslušnosti koncových vrcholov do $S$ a $X$ máme 6 typov hrán, ktoré prispievajú do $A$.
  V nasledujúcom obrázku je pre každý typ hrán uvedená násobnosť, t.j. napríklad každá z hrán medzi $S-X$ a $X-S$
  je zarátaná dvakrát (raz v $\partial(S)$ a raz v $\partial(X)$), každá z hrán medzi $S-X$ a $S\cap X$ 
  sa zarátava raz (v $\partial(X)$)  atď.

  \begin{myfig}{0.4\textwidth}{svg/kcut6}
  \end{myfig}

  \vspace*{-6ex}
  Rozlíšme teraz dva prípady: 

  {\bf 1. $r\in X$.} Budeme uvažovať výraz $B:=\partial(S-X)+\partial(X-S)$ a podobne ako pre výraz $A$, 
  aj teraz sa pozrime, ktoré hrany sa koľkokrát započítajú:
  
  \begin{myfig}{0.4\textwidth}{svg/kcut7}
  \end{myfig}
  \vspace*{-6ex}

  Porovnaním počtov zarátaných hrán vidíme, že $B\le A$, t.j.
  $$\partial(S-X)+\partial(X-S)\le\partial(S)+\partial(X).$$
  Zároveň, pretože $s\in S$,  $X-S$ je $r-s$ rez, a keďže
  $S$ je minimálny $r-s$ rez, platí $\partial(X-S)\ge\partial(S)$.
  Odtiaľ potom dostávame $\partial(S-X)\le\partial(X)$. Lenže $X$ aj $S-X$ sú $v-w$ rezy, a navyše $X$ je minimálny
  $v-w$ rez. Preto  $S-X$ je minimálny $v-w$ rez, pre ktorý platí $S-X\subset S$.


  \vskip 1ex
  {\bf 2. $r\not\in X$.} Postup bude analogický ako v predchádzajúcom prípade, iba teraz budeme uvažovať výraz
  $B':=\partial(S\cup X)+\partial(S\cap X)$.
  
  \begin{myfig}{0.4\textwidth}{svg/kcut8}
  \end{myfig}

  \vspace*{-6ex}
  Porovnaním opäť vidíme $B'\le A$, t.j.
  $$\partial(S\cup X)+\partial(S\cap X)\le\partial(S)+\partial(X).$$
  Pretože $r\not\in S\cup X$, je $S\cup X$ $r-s$ rez, a preto $\partial(S\cup X)\ge\partial(S)$. 
  Z predchádzajúcej nerovnosti potom máme $\partial(S\cap X)\le\partial(X)$, takže $S\cap X$ je minimálny
  $v-w$ rez obsiahnutý v $S$.
\end{dokaz}

Majme teraz jednu iteráciu Gomory-Hu algoritmu, ktorá rozdelila krabicu $S$ na krabice $S_x$ a $S_y$.
Ukážeme, že pre novú hranu $e'=(S_x,S_y)$ platí invariant. Rez definovaný hranou $e'$ je rez, ktorý vznikol
z minimálneho $x-y$ rezu v $G'$ a $\omega'(e')$ je jeho cena. Stačí nám teda ukázať, že minimálny $x-y$ rez v $G'$
je zároveň (po expandovaní vrcholov $y_i$) minimálny $x-y$ rez v $G$; inými slovami, minimálny $x-y$ rez v $G$
nerozdelí vrcholy patriace do jedného podstromu $T_i$.

Nech $K_1,\ldots,K_z$ sú synovia $S$ (t.j. korene podstromov $T_1,\ldots,T_z$) a 
nech $a_i\in K_i$, $s_i\in S$ sú príslušní svedkovia. 
Hrana $(S,K_1)$ definuje rez v $G$, 
v ktorom na jednej strane sú vrcholy podstromu $T_1$, na druhej strane všetky ostatné vrcholy, vrátane $x$ a $y$.
Pretože tento rez je zároveň minimálny $a_1-s_1$ rez, z Lemy~\ref{lm:kcutlema}
dostaneme, že v $G$ existuje $x-y$ rez, ktorý nerozdelí vrcholy podstromu $T_1$. 
Iterovaním tohto argumentu pre ostatné podstromy dostaneme požadované tvrdenie.

\begin{minipage}[t]{0.3\textwidth}
  \vskip 0pt
  \begin{myfig}{\textwidth}{svg/kcut9}
  \end{myfig}
\end{minipage}\hfill
\begin{minipage}[t]{.6\textwidth}
  \vskip 0pt
Posledná vec, ktorú potrebujeme ukázať, sa týka hrán $(S,K_i)$. Na začiatku iterácie boli svedkovia
$a_i\in K_i$, $s_i\in S$, ktorí dosvedčili platnosť invariantu. Bez ujmy na všeobecnosti, nech
po skončení iterácie sa z hrany $(S,K_i)$ stala hrana $(S_x,K_i)$. Ak $s_i\in S_x$, invariant 
zjavne ostáva v platnosti. Mohlo sa ale stať, že $s_i\in S_y$.
V tom prípade zoberieme za nového svedka $x$ a ukážeme, že $f_G(a_i,x)=f_G(a_i,s_i)$.

Na jednej strane, pretože rez definovaný hranou $(S,K_i)$ je minimálny $a_i,s_i$ rez v $G$, a zároveň
oddeľuje $a_i$ od $x$, platí $f_G(a_i,x)\le f_G(a_i,s_i)$.
\end{minipage}

Aby sme ukázali opačnú nerovnosť, vyrobme z $G$  pomocný graf $\hat{G}$, 
v ktorom skontrahujeme $S_y$ do jedného vrchola $\hat{y}$. Pretože $S_x$ a $S_y$ vznikli podľa minimálneho $x-y$ rezu,
môžme použiť Lemu~\ref{lm:kcutlema} a argumentovať, že $f_G(a_i,x)=f_{\hat{G}}(a_i,x)$.
Zároveň z Lemy~\ref{lm:kcutreclema} vieme, že $f_{\hat{G}}(a_i,x)\ge\min\{f_{\hat{G}}(x,\hat{y}), 
f_{\hat{G}}(a_i,\hat{y})\}$.
Z minimálneho $a_i-\hat{y}$ rezu v $\hat{G}$ expandovaním $\hat{y}$ vznikne nejaký  $a_i-s_i$ rez v $G$, a preto
$f_{\hat{G}}(a_i,\hat{y})\ge f_G(a_i,s_i)$.
Z rovnakých dôvodov platí $f_{\hat{G}}(x,\hat{y})\ge f_G(x,y)$; zároveň ale minimálny $x-y$ rez v $G$
oddeľuje $a_i$ a $s_i$, preto $f_G(x,y)\ge f_G(a_i,s_i)$, takže aj $f_{\hat{G}}(x,\hat{y})\ge f_G(a_i,s_i)$.

\subsection*{Naspäť k \kcut}

Uvažujme nasledovný jednoduchý algoritmus: pre daný graf $G=(V,E)$ zostroj Gomory-Hu strom $T$.
Zober $k-1$ najlacnejších hrán z $T$, $T$ sa tak rozpadne na $k$ súvislých komponentov. Tieto
komponenty vráť ako riešenie problému \kcut.
Ukážeme, že tento algoritmus je $2\left(1-\frac{1}{k}\right)$-aproximačný.

Nech hrany $T$ majú váhy $\omega'(e'_1)\le\omega'(e'_2)\le\cdots\le\omega'(e'_{n-1})$ a nech
$C_1,\ldots,C_k$ sú komponenty súvislosti optimálneho riešenia, pričom 
$\partial(C_1)\le\partial(C_2)\le\cdots\le\partial(C_k)$.
Z definície Gomory-Hu stromu, $\omega'(e'_i)=\partial_G(cut_T(e'))$, t.j. cena hrany v $T$ je cena
príslušného rezu v $G$. Keďže
algoritmus zoberie zjednotenie prvých $k-1$ rezov, môže sa stať, že niektoré hrany patria do viacerých
rezov, ale v každom prípade pre cenu algoritmu platí $ALG\le\sum_{i=1}^{k-1}\omega'(e'_i)$.
Na druhej strane, pre cenu optimálneho riešenia platí $2\cdot OPT=\sum_{i=1}^k\partial_G(C_i)$. Naším 
cieľom bude nájsť nejakých $k-1$ hrán v $T$ tak, že $i$-ta nájdená hrana má cenu nanajvýš $\partial_G(C_i)$.
Ak sa nám to podarí, budeme vedieť, že $ALG\le\sum_{i=1}^{k-1}\partial_G(C_i)$, lebo algoritmus
vyberá $k-1$ najlacnejších hrán z $T$. Zvyšné argumenty sú potom rovnaké ako v prípade \multiwaycut.

Zoberme si strom $T$ a skontrahujme každý komponent $C_i$ do jedného vrchola. Takto vzniknutý graf $T'$ môže mať 
cykly aj násobné hrany.

\begin{myfig}{0.9\textwidth}{svg/kcut10}
\end{myfig}

Vyhoďme z $T'$  hrany tak, aby ostal strom $T''$  a zakoreňme ho v $C_k$.
Pre každý vrchol $C_i$, $i=1,\ldots,k-1$ zoberme hranu $e'_{h_i}$, 
ktorá ide k jeho rodičovi v $T''$; dostali sme $k-1$ hrán z $T$. Ukážeme, že $\omega'(e'_{h_i})\le\partial_G(C_i)$,
čím sa celý dôkaz skončí.
Hrana $e'_{h_i}=(u,v)$ má jeden koncový vrchol v $C_i$ a druhý mimo $C_i$. $\omega'(e'_{h_i})$ je cena
minimálneho $u-v$ rezu v $G$. Ale $C_i$ tiež oddeľuje $u$ od $v$, a preto $\omega'(e'_{h_i})\le\partial_G(C_i)$.


