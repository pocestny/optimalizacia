\newpage
\section{\maxflow--\mincut in the view of LP duality}

\noindent 
The LP duality, introduced in the previous section, often helps to better (or at least differently)
understand min-max characterizations of computational problems. In this part we employ the duality 
theory to give an alternative proof of the known fact that the size of the maximum $s-t$ flow equals to 
the capacity of the minimum $s-t$ cut. Since we suppose that the reader is already familiar with the 
\maxflow and \mincut problems, we abstain from motivational examples, and just recall the definitions.

\noindent
Given is an (undirected) graph $G=(V,E)$ with non-negative edge weights, i.e.
a mapping $c:E\mapsto\R^+$; instead of  $c(\{u,v\})$ we shall use a shorthand notation
$c_{uv}=c_{vu}$. In the graph, there are two distinguished vertices $a$ and $t$,
The \maxflow problem aims to maximize the flow from $s$ to $t$, where a flow is a mapping
$f:V^2\mapsto\R^+$, such that $f(u,v)$ represents the amount of a liquid that flows
from $u$ to $v$ in an unit time. The flow must hence satisfy the following properties:
\begin{enumerate}
  \item if $f(u,v)\not=0$ then  $(u,v)\in E$, i.e. the flow can be only along the edges of the graph
  \item $f(u,v)=-f(v,u)$, i.e. the sign gives the direction of the flow
  \item for each $v\not\in\{s,t\}$ it holds $\sum\limits_{u\in V}f(u,v)=0$, 
  known as a flow conservation law
  \item $|f(u,v)|\le c_{uv}$, i.e. the flow is bounded by the capacity of the edge
\end{enumerate}
\noindent
The size of the flow is  $\sum_{v\in V}f(s,v)$. 
A cut in $G$ is a set of edges in $G$ whose removal disconnects $s$, and $t$ (i.e. they are in different
connected components of the resulting graph). The \mincut problems asks to find a cut with minimal capacity.
The following figure depicts a graph with a minimum cut (left), and a maximum flow (right; the dark edges 
are those whose capacity is fully used by the flow) with value 24.


\begin{myfig}{\textwidth}{svg/flowcutNew}
\end{myfig}


\noindent
The \maxflow problem has a number of  natural formulations as a linear program. For reasons we hope
to become apparent by the end of this section we choose the following one: for each edge $(u,v)\in E$ 
we introduce two non-negative variables $x_{uv}$ and $x_{vu}$ that denote the amount of flow
along the edge $(u,v)$ (note that, by doing this, we allow to have a non-negative amount of flow $x_{uv}$ from
$u$ to $v$, and at the same time, some non-negative amount of flow $x_{vu}$ from $v$ to $u$; however,
it is easy to verify that simply subtracting the two values maintains the feasibility and value of a solution,
and makes it compatible with our definition of flow).  Next, we introduce a variable $f$ that will
denote the amount of the whole flow (and the goal will be to maximize $f$):
$f=\sum\limits_{u:(s,u)\in E}x_{su}-\sum\limits_{u:(s,u)\in E}x_{us}$. 
Flow conservation law, and the capacity constraints can easily be expressed as linear inequalities, yielding
the following program:

\begin{equation}
  \label{eq:flow:1}
  \begin{array}{rrcll}
      {\rm maximize}     & \multicolumn{1}{r}{f}\\[8mm]
    {\rm subject\ to} & \sum\limits_{u:(s,u)\in E} x_{su} - \sum\limits_{u:(s,u)\in E} x_{us} - f  &=&0&  \\[6mm]
                            & \sum\limits_{u:(t,u)\in E} x_{ut} - \sum\limits_{u:(s,u)\in E} x_{tu} + f  &=&0&  \\[6mm]
                            & \sum\limits_{u:(u,v)\in E} x_{vu} - \sum\limits_{u:(u,v)\in E} x_{uv} &=&0&\;\;\;
    \forall v\in V - \{s,t\}\\[6mm]
    & x_{uv}&\le& c_{uv}&  \;\;\;\forall (u,v)\in E\\[1mm]
    & x_{vu}&\le& c_{uv}&  \;\;\;\forall (u,v)\in E\\[1mm]
                            & x_{uv}&\ge& 0 &  \;\;\;\forall (u,v)\in E\\
  \end{array}
\end{equation}  

\noindent
The first two constraints define the size of the flow $f$ as the net amount that leaves $s$, or enters $t$,
respectively. All remaining vertices (the next set of constraints) obey the flow conservation law. 

\noindent
Now let us apply our dualization recipe, and look at the dual minimization program to the program (\ref{eq:flow:1})
Every constraint in the primal program gives rise to a variable in the dual program, In the program  (\ref{eq:flow:1})
there are two types of constraints: the constraints of the form $\cdots = 0$ for each vertex, and
constraints  $x_{uv}\le c_{uv}$. Introduce a dual variable  $y_v\in\R$ for each vertex $v\in V$
($y_s$ corresponds to the first constraint, $y_t$ to the second one, and the rest to the remaining vertices),
and a pair of non-negative variables $z_{uv}$ a $z_{vu}$ for each edge $(u,e)\in E$. The $i$-th dual variable
is a multiplier of the $i$-th constraint in the linear combination with the requirement that sum of the left-hand
sides is component-wise bigger than the maximized utility function. The sum of the right-hand sides is in 
our case $\sum\limits_{(u,v)\in E}(z_{uv}+z_{vu})c_{uv}$.
The maximized function is zero in all components but $f$. The variable $f$ appears only in the first and 
second constraints of  (\ref{eq:flow:1}), hence we get in the dual a constraint $y_t-y_s=1$
(note that although in the optimum solution $f$ clearly is non-negative, we did not require the variable $f\ge0$).
Every variable $x_{uv}$ appears in three constraints: the constrain corresponding to vertex the $u$ with positive sign,
the constraint corresponding to the vertex $v$ with negative sign, and in the constraint $x_{uv}\le c_{uv}$. 
Summarizing, we get the following program:

\begin{equation}
  \label{eq:flow:2}
  \begin{array}{rrcll}
    {\rm minimize}     & \multicolumn{1}{r}{\sum\limits_{(u,v)\in E}(z_{uv}+z_{vu})\,c_{uv}}\\[8mm]
    {\rm subject\ to} & y_t - y_s  &=&1&  \\[3mm]
                            & y_v - y_u + z_{uv} &\ge&0&\;\;\;    \forall (u,v)\in E \\[3mm]
                            & y_u - y_v + z_{vu} &\ge&0&\;\;\;    \forall (u,v)\in E \\[3mm]
                            & z_{uv},z_{vu}&\ge& 0 &  \;\;\;\forall (u,v)\in E\\
  \end{array}
\end{equation}  

\noindent
From the strong duality theorem we know that programs (\ref{eq:flow:1}) and (\ref{eq:flow:2})
have the same value of the optimum. How can we interpret the  program  (\ref{eq:flow:2})?
Without loss of generality we can suppose that at least one of the pair of variables  $z_{uv},z_{vu}$
is zero: the only constraints concerning  $z_{uv}$ or $z_{vu}$ are the constraints
$z_{uv}\ge y_u-y_v$ and $z_{vu}\ge y_v-y_u$. Clearly, at least one of the values 
 $y_u-y_v$ and  $y_v-y_u$ is not positive, and so we can set at least one of them to zero in any feasible solution,
and not increase the value of the utility function. Hence, we can denote $\bar{z}_{uv}=\max\{z_{uv},z_{vu}\}$;
the constraints then ensure that $\bar{z}_{uv}\ge|y_u-y_v|$ and using the same arguments as before we can
suppose without loss of generality  $\bar{z}_{uv}=|y_u-y_v|$.
Now we can see the  solving of the program as placing the vertices of the graph on a line: the vertex $v$
is placed in the point $y_v$, where again without loss of generality we may assume that $s$ is placed in $0$, and
$t$ in $1$. The value  $\bar{z}_{uv}$ is the length of the edge $(u,v)$ in this placement of vertices.
Altogether, the optimal solution of the program  (\ref{eq:flow:2}) places the vertices of the graph on a line
segment of length 1 in such a way that $s$ and $t$ are on the endpoints, and the overall length 
of edges, weighted by their capacity, is minimized.

\begin{myfig}{0.9\textwidth}{svg/relaxcut}
  A graph with edge weights (left) and one of possible optimal solutions (right) of the program
  (\ref{eq:flow:2}):
  $\bar{z}_{su}=\bar{z}_{vw}=\frac{1}{6}$, $\bar{z}_{sv}=\bar{z}_{uw}=\frac{1}{3}$,
  $\bar{z}_{sw}=\bar{z}_{wt}=\frac{1}{2}$. The resulting value is
  $$\sum_{(u,v)\in E}c_{uv}\bar{z}_{uv}=\frac{1}{6}+\frac{1}{3}+2\frac{1}{2}+\frac{1}{3}+\frac{1}{6}+4\frac{1}{2}=4.$$
\end{myfig}

\noindent
Now let us rewrite the program  (\ref{eq:flow:2}) in normal form 
$\max\{-\bm{c}\tr\bm{\beta}\mid A\bm{\beta}=0,\;\bm{\beta}\ge0\}$.
Introduce a slackness variable $\hat{z}_{uv}\ge 0$ to each variable  $z_{uv}$, obtaining the
constraints of the form  \hbox{$y_v-y_u+z_{uv}-\hat{z}_{uv}=0$}.
If the vector  $\bm{\beta}$ contains the values ordered as  
$y_s,y_t,y_{v_1},\ldots,z_{uv},z_{vu},\ldots,\hat{z}_{uv}.\hat{z}_{vu},\ldots$, 
the matrix $A$ has the following structure:

\begin{myfig}{0.8\textwidth}{svg/cutlp}
\end{myfig}

\noindent
The following straightforward chores are left to the reader:


\begin{prob}
  Using Theorems \ref{thm:idTum} and \ref{thm:unimod} prove that the matrix $A$ is TUM.
\end{prob}


\noindent
Theorem~\ref{thm:tumInteger} implies that there exists an integral optimal solution of the 
program (\ref{eq:flow:2}). It means that all vertices have $y_v\in\{0,1\}$, i.e. they are placed in one of 
the endpoints of the line, and each edge has length 0 or 1. That means that every $s-t$ path must contain at least
one edge with length 1, which implies that the removal of edges with length 1 disconnects $s$ and $t$.
So we can say

\begin{veta}[\maxflow-\mincut theorem]
  The size of the maximal flow is equal to the capacity of the minimal cut.
\end{veta}
\begin{dokaz}
  The size of the maximal flow is the optimum value of the program (\ref{eq:flow:1}), which is the same
  as the optimum of the program  (\ref{eq:flow:2}). There exists an integral optimum of (\ref{eq:flow:2})
  and at the same time there is a bijection between $s-t$ cuts and integral solutions of  (\ref{eq:flow:2}).
\end{dokaz}

\noindent
It is quite possible that the reader is familiar with a proof that is significantly simpler, and does not involve
the linear programming. The reason why we present this argument (apart from the fact that we want to 
give a more concrete  example of the LP duality) is that the \maxflow-\mincut pair is seen as a special case
of a general class of problems; at the same time, it sheds a new light into the question why a similar result 
does not hold for seemingly similar problems (e.g. we would have more sources and sinks, and the goal would be
to maximize the flow of several commodities in common networks of pipes, we would get a problem that is dual to the
relaxed version of the \minmulticut from Definition~\ref{dfn:multicut}, however, a similar result does not hold)
via the unimodularity of the corresponding matrices.
